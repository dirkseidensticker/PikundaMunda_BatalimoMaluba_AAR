An unfortunate but common practice in order to 'date' nodes in linguistic reconstructions of modern Bantu languages is 'calibrating' them using archaeological data: for example, \citet[SI p.~2]{Grollemund.2015} equates an initial off-branching in the Cameroon-Nigeria homeland with the archaeological inventories of the lower horizon of the Gray Ash layer from the site of Shum Laka, representing the second phase of the "Stone to Metal Age" (SMA) or "Ceramic Late Stone Age" and dating into the 3rd to 2nd millennium BCE \citep[226--231,243]{Lavachery.2001}. During that phase the microlithic quartz industry, dominating the late Pleistocene deposits \citep[172 Fig.~4]{Cornelissen.2003,Cornelissen.2017}, is slowly declining, while a macrolithic flake and blade industry on basalt, which occurs in smaller numbers in older deposits as well, becomes more prevailing \citep[169 Fig.~1]{Cornelissen.2003,Cornelissen.2017}. Pottery, while already present in the older deposits of the upper horizon of the Ochre Ash layer and dating into the 5th to 4th millennium BCE \citep[224-225 Fig.~4.2--3]{Lavachery.2001}, is now more prominent and decorated with grooves and impressions, including rocker zig-zag \citep[231--232 Fig.~8]{Lavachery.2001}. It must be stressed that Shum Laka represents the only site in the wider region of the putative homeland of the Bantu languages, in the border area of Nigeria and Cameroon, that has been studied in detail. Thus the deciding factors of \citet{Grollemund.2015} for 'selecting' the specific deposits at Shum Laka as an archaeological 'calibration point' for their phylogentic language tree is neither quantitative nor qualitative. And while \citet[SI p.~2]{Grollemund.2015} allege that "small immigrant communities from further north" introduced pottery and Benue-Congo languages, more recent genetic analyses from four burials at Shum Laka showed that the genetic profiles of the occupants of the site "are very different from those of most speakers of Niger–Congo languages today, which implies that these individuals are not representative of the primary source population(s) that were ancestral to present-day Bantu-speakers" \citep[5]{Lipson.2020}. 

\citet[SI p.~2]{Grollemund.2015} date the subsequent node of Bantu languages in the younger half of the 2nd millennium BCE based on the earliest occurrence of markers for 'sedentism' at Obobogo in southern-central Cameroon \citep{deMaret.1982,deMaret.1983,deMaret.1992}. It must be noted that final analyses from the site of Obobogo are still pending. \citet{Claes.1985} provided some initial results, but not a consecutive analysis of all available data. \citet[SI p.~2]{Grollemund.2015} associate the third mayor branching-off point with the emergence for Urewe ceramics in the Great Lakes region of East Africa.

All these choices together perpetuated the trope that early Bantu-speakers can be equated with the earliest pottery production in a given region \citep[355,362,364]{Bostoen.2015}. This generalization further suffers from disregarding the (dis-)continuities of a single facet of material culture in a given region and their relations to any historically identifiable human society. In a nutshell, the procedure of adopting opportune archaeological results for underpinning linguistic reconstructions by \citet{Grollemund.2015} and \citet{Bostoen.2015}, which was subsequently adopted by \citet[SI]{Koile.2022} without any critical review, represents another facet of a long-standing tradition in circular reasoning \citep{Ehret.1973,Phillipson.1976,Phillipson.1976b,Phillipson.1977a,Heine.1977} that has been reviewed by \citet[82]{Eggert.2005,Eggert.2016a}. Such approaches induce 'procedural puzzles' and fail at linking linguistics with "the authentic material evidence of archaeology" \citep[88]{Eggert.2016a}. In consequence it can only be reiterated that "non-written languages do not leave material traces" \citep[85]{Eggert.2016a}.